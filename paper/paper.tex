\documentclass{sigchi}

% pandoc setup %



\usepackage[backend=biber]{biblatex}
\addbibresource{references.bib}

\providecommand{\tightlist}{%
  \setlength{\itemsep}{0pt}\setlength{\parskip}{0pt}}

% end pandoc setup %

% Use this section to set the ACM copyright statement (e.g. for
% preprints).  Consult the conference website for the camera-ready
% copyright statement.

% Copyright
\CopyrightYear{2020}
%\setcopyright{acmcopyright}
\setcopyright{acmlicensed}
%\setcopyright{rightsretained}
%\setcopyright{usgov}
%\setcopyright{usgovmixed}
%\setcopyright{cagov}
%\setcopyright{cagovmixed}
% DOI
\doi{https://doi.org/10.1145/3313831.XXXXXXX}
% ISBN
\isbn{978-1-4503-6708-0/20/04}
%Conference
\conferenceinfo{CHI'20,}{April  25--30, 2020, Honolulu, HI, USA}
%Price
\acmPrice{\$15.00}

% Use this command to override the default ACM copyright statement
% (e.g. for preprints).  Consult the conference website for the
% camera-ready copyright statement.

%% HOW TO OVERRIDE THE DEFAULT COPYRIGHT STRIP --
%% Please note you need to make sure the copy for your specific
%% license is used here!
% \toappear{
% Permission to make digital or hard copies of all or part of this work
% for personal or classroom use is granted without fee provided that
% copies are not made or distributed for profit or commercial advantage
% and that copies bear this notice and the full citation on the first
% page. Copyrights for components of this work owned by others than ACM
% must be honored. Abstracting with credit is permitted. To copy
% otherwise, or republish, to post on servers or to redistribute to
% lists, requires prior specific permission and/or a fee. Request
% permissions from \href{mailto:Permissions@acm.org}{Permissions@acm.org}. \\
% \emph{CHI '16},  May 07--12, 2016, San Jose, CA, USA \\
% ACM xxx-x-xxxx-xxxx-x/xx/xx\ldots \$15.00 \\
% DOI: \url{http://dx.doi.org/xx.xxxx/xxxxxxx.xxxxxxx}
% }

% Arabic page numbers for submission.  Remove this line to eliminate
% page numbers for the camera ready copy
% \pagenumbering{arabic}

% Load basic packages
\usepackage{balance}       % to better equalize the last page
\usepackage{graphics}      % for EPS, load graphicx instead
\usepackage[T1]{fontenc}   % for umlauts and other diaeresis
\usepackage{txfonts}
\usepackage{mathptmx}
\usepackage[pdflang={en-US},pdftex]{hyperref}
\usepackage{color}
\usepackage{booktabs}
\usepackage{textcomp}

% Some optional stuff you might like/need.
\usepackage{microtype}        % Improved Tracking and Kerning
% \usepackage[all]{hypcap}    % Fixes bug in hyperref caption linking
\usepackage{ccicons}          % Cite your images correctly!
% \usepackage[utf8]{inputenc} % for a UTF8 editor only

% If you want to use todo notes, marginpars etc. during creation of
% your draft document, you have to enable the "chi_draft" option for
% the document class. To do this, change the very first line to:
% "\documentclass[chi_draft]{sigchi}". You can then place todo notes
% by using the "\todo{...}"  command. Make sure to disable the draft
% option again before submitting your final document.
\usepackage{todonotes}

% Paper metadata (use plain text, for PDF inclusion and later
% re-using, if desired).  Use \emtpyauthor when submitting for review
% so you remain anonymous.
\def\plaintitle{Abstract Program Visualization for Model-View-Update
User Interfaces}
\def\plainauthor{Geoffrey Litt}
\def\emptyauthor{}
\def\plainkeywords{Program visualization, program understanding, debugging}
\def\plaingeneralterms{Program visualization, program understanding, debugging}

% llt: Define a global style for URLs, rather that the default one
\makeatletter
\def\url@leostyle{%
  \@ifundefined{selectfont}{
    \def\UrlFont{\sf}
  }{
    \def\UrlFont{\small\bf\ttfamily}
  }}
\makeatother
\urlstyle{leo}

% To make various LaTeX processors do the right thing with page size.
\def\pprw{8.5in}
\def\pprh{11in}
\special{papersize=\pprw,\pprh}
\setlength{\paperwidth}{\pprw}
\setlength{\paperheight}{\pprh}
\setlength{\pdfpagewidth}{\pprw}
\setlength{\pdfpageheight}{\pprh}

% Make sure hyperref comes last of your loaded packages, to give it a
% fighting chance of not being over-written, since its job is to
% redefine many LaTeX commands.
\definecolor{linkColor}{RGB}{6,125,233}
\hypersetup{%
  pdftitle={\plaintitle},
% Use \plainauthor for final version.
%  pdfauthor={\plainauthor},
  pdfauthor={\plainauthor},
  pdfkeywords={\plainkeywords},
  pdfdisplaydoctitle=true, % For Accessibility
  bookmarksnumbered,
  pdfstartview={FitH},
  colorlinks,
  citecolor=black,
  filecolor=black,
  linkcolor=black,
  urlcolor=linkColor,
  breaklinks=true,
  hypertexnames=false
}

% create a shortcut to typeset table headings
% \newcommand\tabhead[1]{\small\textbf{#1}}

% End of preamble. Here it comes the document.
\begin{document}

\title{\plaintitle}

\numberofauthors{1}
\author{%
  \alignauthor{Geoffrey Litt}\\
    \affaddr{MIT CSAIL}\\
    \email{glitt@mit.edu}\\
}

\maketitle

\begin{abstract}
  This is the abstract
\end{abstract}


% ACM Classfication

\begin{CCSXML}
<ccs2012>
<concept>
<concept_id>10003120.10003121</concept_id>
<concept_desc>Human-centered computing~Human computer interaction (HCI)</concept_desc>
<concept_significance>500</concept_significance>
</concept>
<concept>
<concept_id>10003120.10003121.10003125.10011752</concept_id>
<concept_desc>Human-centered computing~Haptic devices</concept_desc>
<concept_significance>300</concept_significance>
</concept>
<concept>
<concept_id>10003120.10003121.10003122.10003334</concept_id>
<concept_desc>Human-centered computing~User studies</concept_desc>
<concept_significance>100</concept_significance>
</concept>
</ccs2012>
\end{CCSXML}

\ccsdesc[500]{Human-centered computing~Human computer interaction (HCI)}
\ccsdesc[300]{Human-centered computing~Haptic devices}
\ccsdesc[100]{Human-centered computing~User studies}

% Author Keywords
\keywords{\plainkeywords}

% Print the classficiation codes
\printccsdesc
Please use the 2012 Classifiers and see this link to embed them in the text: \url{https://dl.acm.org/ccs/ccs_flat.cfm}

\hypertarget{introduction}{%
\section{Introduction}\label{introduction}}

Lots of recent program vis work is low level. Tied directly to the
source code, visualizing state at individual lines, individual
variables. Makes sense for novices: small programs, need help
understanding small details.

OTOH: many tasks require higher level of understanding system behavior.
Eg, initial explanation of a codebase. How do we make a ``whiteboard
drawing'' live? Some challenges:

\begin{itemize}
\tightlist
\item
  relevant state isn't necessarily individual variable values: need to
  select certain relevant attributes to visualize. Corollary: can't be
  as automated as low level visualization.
\item
  Too much work on ``targeted debugging'', not enough on generalized
  understanding of a system. More specifically: if we don't know what
  change you might want to make to a codebase, how can we maximally
  equip you to be ready to make an arbitrary change?

  \begin{itemize}
  \tightlist
  \item
    Naur Programming as theory building
  \end{itemize}
\item
  combines both static + dynamic aspects: code modules, abstracted state
\item
  Lots of non-numeric data
\end{itemize}

Abstract visualization is an interesting direction. Can be
super-specific, eg algorithm animation \autocite{brown1984,stasko1990}
or class and thread vis of \autocite{reiss2003,reiss2005}.

But a major challenge is making it easy enough for programmer to
actually make it worth it \autocite{reiss2007}

This is an initial exploration. Some solution characteristics:

\begin{itemize}
\tightlist
\item
  redux apps \autocite{czaplicki,fowler2020}: 1) broader than Vega,
  narrower than general programs, 2) imposes a worldview of abstract
  state and simplified event stream. THIS IS THE HEART OF THE WORK

  \begin{itemize}
  \tightlist
  \item
    In this work: focus on TodoMVC specifically, but in theory should
    generalize to Redux programs. (Future work: how to efficiently
    generate a visualization like this one)
  \end{itemize}
\item
  ``Guided tour'': introducing you to how the application works.
\end{itemize}

\hypertarget{sec:related-work}{%
\section{Related Work}\label{sec:related-work}}

\begin{itemize}
\tightlist
\item
  low level program vis

  \begin{itemize}
  \tightlist
  \item
    Learnable \autocite{victora}
  \item
    omnicode \autocite{kang2017}
  \item
    Projection boxes \autocite{lerner2020}
  \item
    Theia \autocite{pollock2019}
  \item
    Theseus \autocite{lieber2014}
  \end{itemize}
\item
  Whyline, targeted interrogation \autocite{ko2004}
\item
  In-situ: nice taxonomy of visualizations (but still limited to Vega,
  narrow domain) \autocite{hoffswell2018a}
\item
  Myers taxonomy \autocite{myers1990}
\item
  Redux / Elm

  \begin{itemize}
  \tightlist
  \item
    redux dev tools Tree viewer
  \end{itemize}
\item
  http://cs.brown.edu/\textasciitilde spr/research/bloom/jvlexec.pdf
\item
  Steve Reiss overview
\item
  Brad Myers incense
\item
  Girba's Mondrian: a toolkit for programmers building vis
  \autocite{meyer2006}
\end{itemize}

\hypertarget{methods}{%
\subsection{Methods}\label{methods}}

Only some initial experiments.

Redux dev tools monitor D3 + react

Show the mockups Developed 3-4 sparkline style visualizations

\begin{itemize}
\tightlist
\item
  line graph
\item
  Categorical graph
\item
  Collection dots view?
\item
  String change view?
\end{itemize}

\hypertarget{results}{%
\subsection{Results}\label{results}}

\begin{itemize}
\tightlist
\item
  get feedback from 1-2 people?
\item
  Tried using it myself
\end{itemize}

\hypertarget{discussion}{%
\subsection{Discussion}\label{discussion}}

\hypertarget{future-work}{%
\subsection*{Future work}\label{future-work}}
\addcontentsline{toc}{subsection}{Future work}

% Balancing columns in a ref list is a bit of a pain because you
% either use a hack like flushend or balance, or manually insert
% a column break.  http://www.tex.ac.uk/cgi-bin/texfaq2html?label=balance
% multicols doesn't work because we're already in two-column mode,
% and flushend isn't awesome, so I choose balance.  See this
% for more info: http://cs.brown.edu/system/software/latex/doc/balance.pdf
%
% Note that in a perfect world balance wants to be in the first
% column of the last page.
%
% If balance doesn't work for you, you can remove that and
% hard-code a column break into the bbl file right before you
% submit:
%
% http://stackoverflow.com/questions/2149854/how-to-manually-equalize-columns-
% in-an-ieee-paper-if-using-bibtex
%
% Or, just remove \balance and give up on balancing the last page.
%
\balance{}

% \bibliographystyle{SIGCHI-Reference-Format}
% \bibliography{sample}

\printbibliography

\end{document}

%%% Local Variables:
%%% mode: latex
%%% TeX-master: t
%%% End:
